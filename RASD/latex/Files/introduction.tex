\renewcommand{\arraystretch}{1.5} % Adjust the multiplier as needed

\section{Purpose}
\label{s:Purpose}%

The CodeKata is a learning method that takes inspiration from the Kata techniques and is based on continuous practice which became very popular in those years.

CodeKataBattle delineates an innovative platform geared towards enhancing students' software development skills through collaborative learning using CodeKata’s fundamentals. Facilitated by educators, CKB provides a dynamic environment where students engage in code kata battles, refining their programming proficiency and embracing best practices such as the test-driven development approach.

CKB empowers educators to orchestrate challenges within competition, stimulating healthy competition and cultivating an environment for skill enhancement. The platform enables educators to define battle parameters, set deadlines, and configure scoring criteria, fostering a tailored and effective learning experience.

At its core, a code kata battle presents students with programming challenges within specific language frameworks, coupled with exhaustive test cases. Teams collaboratively tackle these exercises, adhering to a test-first methodology and submitting solutions to the platform upon battle completion.

CKB's automated evaluation system ensures an impartial assessment of student submissions. Automated scrutiny covers mandatory factors, including functional aspects, timeliness, and source code quality, offering an unbiased representation of team performance. Educators can further enhance evaluations with optional manual assessments, providing nuanced insights into student work.

\pagebreak

\subsection{Goals}
\label{ss:goals}%
The platform will be used by two types of users: Educators (ED) and Students (ST). The ED will be able to create competitions and battles within competitions. The ST will be able to create teams and join battles as a team or individually. The platform will communicate with Github to pull the latest code submission and provides automated evaluation of the code submitted. The platform will also show a ranking of the competition and battles.

Below there is the table of goals that the platform will achieve:
\begin{longtable}{|l|l|}
  \hline
  \textbf{\#} & \textbf{Goal}      \\
  \hline
  G1 & Enable ED to manage competitions \\
  \hline
  G2 & Enable ED to manage code battles within competitions \\
  \hline
  G3 & Enable ST to participate in a competition \\
  \hline
  G4 & Enable ST to be part of a team within a battle \\
  \hline
  G5 & Send Notifications to STs   \\
  \hline
  G6 & Automatically create GitHub repositories for every battle in a competition    \\
  \hline
  G7 & Synchronize the submission of each candidate with their GitHub repository   \\
  \hline
  G8 & CKB provides an evaluation of the code submitted    \\
  \hline
  G9 & Allow users to view rankings in both battles and competitions   \\
  \hline
  G10 & Allow to assign badges to the students    \\
  \hline
  G11 & Allow users to visualize ST profiles \\
  \hline
  \caption{List of goals}
  \label{tab:goals}
\end{longtable}

\pagebreak
\section{Scope}
\label{s:Scope}%

\subsection{World phenomena}
\label{ss:world_phenomena}%

\begin{table}[H]
  \begin{tabular}{|l|l|}

    \hline
    \textbf{ID} & \textbf{Definitions}      \\
    \hline
    WP1 & ED wants to create a competitions \\
    \hline
    WP2 & ED wants to create a battle \\
    \hline
    WP3 & ST wants to participate in a competition \\
    \hline
    WP4 & ST wants to participate in a battle     \\
    \hline
    WP5 & ST set up GitHub actions    \\
    \hline
    
  \end{tabular}
  \caption{List of the world phenomena}
  \label{tab:worldPhenomena}
\end{table}

\subsection{Shared phenomena}
\label{ss:shared_phenomena}%

\begin{longtable}{|l|l|}

  \hline
  \textbf{ID} & \textbf{Definitions}      \\
  \hline
  SP1 & ST creates an account on the platform \\
  \hline
  SP2 & ED creates an account in the platform \\
  \hline
  SP3 & ST logs in to the platform \\
  \hline
  SP4 & ED logs in to the platform  \\
  \hline
  SP5 & ST registers for the competitions before the deadline   \\
  \hline
  SP6 & ED creates a badge with certain rules   \\
  \hline
  SP7 & ED manually evaluates the code submitted by students   \\
  \hline
  SP8 & ED creates a competition   \\
  \hline
  SP9 & ED creates a battle within a competition   \\
  \hline
  SP10 & ED closes a competition   \\
  \hline
  SP11 & ST pushes new commit(s) into their GitHub repository before the deadline   \\
  \hline
  SP12 & ST invites other STs to participate in a battle as a team   \\
  \hline
  SP13 & ST subscribes as a single/team for an incoming battle before the deadline   \\
  \hline
  SP14 & CKB sends a notification that a competition is available to ST   \\
  \hline
  SP15 & CKB sends a notification that a battle is created inside a competition to ST   \\
  \hline
  SP16 & CKB sends a notification that a competition has ended to ST   \\
  \hline
  SP17 & CKB sends a notification that a battle has ended to ST   \\
  \hline
  SP18 & CKB sends links to the GitHub repository to all the ST subscribed   \\
  \hline
  SP19 & CKB updates scores for each ST   \\
  \hline
  SP20 & CKB gives badge to ST   \\
  \hline
  SP21 & CKB updates the ranking of the competition   \\
  \hline
  SP22 & CKB updates the ranking of the battle   \\
  \hline
  
  \caption{List of the shared phenomena}
  \label{tab:sharedPhenomena}
\end{longtable}


\section{Definitions, Acronyms, Abbreviations}
\label{s:Definitions_Acronyms_Abbreviations}%

\subsection{Definitions}
\label{ss:Definitions}

\begin{table}[H]
  \begin{tabular}{|l|p{0.8\textwidth}|}

    \hline
    User & Anyone interacting with the system, it can be both a Student or an Educator    \\
    \hline
    Manage & Create, supervise and edit a certain element of the application. \\
    \hline
    Code Kata & A challenge intended to improve programming abilities, including description, test cases and build automation scripts. \\
    \hline
  \end{tabular}
  \caption{List of definitions}
  \label{tab:definitions}
\end{table}

\subsection{Acronyms}
\label{ss:Acronyms}

\begin{table}[H]
  \begin{tabular}{|l|l|}

    \hline
    ST & Student \\
    \hline
    ED & Educator \\
    \hline
    CKB & CodaKataBattle \\
    \hline
    RASD & Requirements Analysis and Specification Document     \\
    \hline
    SAT & Static Analyzer Tool    \\
    \hline
    T & Team    \\
    \hline
  \end{tabular}
  \caption{List of Acronyms}
  \label{tab:acronyms}
\end{table}

\subsection{Abbreviations}
\label{ss:Abbreviations}


\begin{table}[H]
  \begin{tabular}{|l|l|}

    \hline
    WPX & World Phenomena X    \\
    \hline
    SPX & Shared Phenomena X    \\
    \hline
    GX & Goal Number X    \\
    \hline
    DX & Domain Assumption X    \\
    \hline
    UCX & Use Case X    \\
    \hline

  \end{tabular}
  \caption{List of abbreviations}
  \label{tab:abbreviations}
\end{table}



\section{Revision history}
\label{s:Revision_history}%

  \begin{table}[H]
  \begin{tabular}{|l|l|l|}

    \hline
    Date & Revision & Notes    \\
    \hline
    22/12/2023 & v1.0 & first release    \\
    \hline

  \end{tabular}
  \caption{Revision table}
  \label{tab:revision}
\end{table}


\section{Reference Documents}
\label{s:Reference_documents}%

\begin{itemize}
  \item The specification document of the project: \textit{Assignment RDD AY 2023-2024}
  \item Alloy documentation: \url{https://alloytools.org/documentation.html}
  \item IEEE Standard on Requirement Engineering (ISO/IEC/IEEE 29148): \url{https://standards.ieee.org/ieee/29148/6937/}
\end{itemize}
\section{Document Structure}
\label{s:Document_Structure}%

The document is divided in five main section described as below.

The first section is the introduction, that introduce the goals of the project, purpose and the analysis of world and shared phenomena. It also contains the definitions, acronyms and abbreviations used in the document. 

The second section is the overall description, that contains the general factors that affect the product. Here there is also the analysis of the scenarios and functions of the platform and the domain assumptions.

Then as third section there is the specific requirements section, that contains the functional and non-functional requirements of the platform. Morover, there is a more detailed analysis of the use cases and the mapping between goals and requirements. Then there is a description of the interfaces necessary for the platform to implement all the functionalities.

The fourth section is the formal analysis using Alloy. Here there is the description of the model and the world generated by the Alloy Analyzer. This section is very important to prove the correctness of the model described in the previous sections.

The last section is the effort spent by each member of the group to write this document.
