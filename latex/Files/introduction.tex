\renewcommand{\arraystretch}{1.5} % Adjust the multiplier as needed

\section{Purpose}
\label{s:Purpose}%

The CodeKata is a learning method that takes inspiration from the Kata techniques and is based on continuous practice which became very popular in those years.

CodeKataBattle delineates an innovative platform geared towards enhancing students' software development skills through collaborative learning using CodeKata’s fundamentals. Facilitated by educators, CKB provides a dynamic environment where students engage in code kata battles, refining their programming proficiency and embracing best practices such as the test-driven development approach.

Similar to recent initiatives addressing global challenges, CKB empowers educators to orchestrate challenges within tournaments, fostering healthy competition and cultivating an environment for skill enhancement. The platform enables educators to define battle parameters, set deadlines, and configure scoring criteria, fostering a tailored and effective learning experience.

At its core, a code kata battle presents students with programming challenges within specific language frameworks, coupled with exhaustive test cases. Teams collaboratively tackle these exercises, adhering to a test-first methodology and submitting solutions to the platform upon battle completion.

CKB's automated evaluation system ensures an impartial assessment of student submissions. Automated scrutiny covers mandatory factors, including functional aspects, timeliness, and source code quality, offering an unbiased representation of team performance. Educators can further enhance evaluations with optional manual assessments, providing nuanced insights into student work.


\subsection{Goals}
\label{ss:goals}%

\begin{table}[H]
  \begin{tabular}{|l|l|}
    \hline
    \textbf{\#} & \textbf{Goal}      \\
    \hline
    G1 & Enable ED to Create New Competitions \\
    \hline
    G2 & Enable ED to Create Code Battles within Competitions \\
    \hline
    G3 & Enable ST to Create Teams by Inviting Other STs \\
    \hline
    G4 & Enable ST to Join Teams for Which They Have Been Invited    \\
    \hline
    G5 & Allow STs to Join Battles as a Team    \\
    \hline
    G6 & Allow STs to Join Battles Individually    \\
    \hline
    G7 & Send Notifications to STs about New Competitions and Closing of Competitions    \\
    \hline
    G8 & Automatically Create GitHub Repositories for Every Battle in a Competition    \\
    \hline
    G9 & Synchronize the Submission of Each Candidate with Their GitHub Repository    \\
    \hline
    G10 & Provide a Dashboard for Code Submission    \\
    \hline
    G11 & CBK Provides an automated evaluation of the code submitted  \\
    \hline
    G12 & Provide Automated Evaluation of Submitted Code  \\
    \hline
    G13 & Assign Points to STs Based on Code Evaluation \\
    \hline
    G14 & Allow STs to View Rankings of Competition \\
    \hline
    G15 & Allow STs to View Rankings of  battles only in competition for which are subscribed \\
    \hline
  \end{tabular}
  \caption{Goals}
  \label{tab:goals}
\end{table}

\section{Scope}
\label{s:Scope}%


\subsection{World phenomena}
\label{ss:world_phenomena}%

\begin{table}[H]
  \begin{tabular}{|l|l|}

    \hline
    \textbf{\#} & \textbf{World phenomena}      \\
    \hline
    WP1 & ED wants to create a competitions \\
    \hline
    WP2 & ED wants to create a battle \\
    \hline
    WP3 & ST wants to participate in a competition \\
    \hline
    WP4 & ST wants to participate in a battle     \\
    \hline
    WP5 & ST set up GitHub actions    \\
    \hline
  \end{tabular}
  \caption{World phenomena table}
  \label{tab:worldPhenomena}
\end{table}


\section{Definitions, Acronyms, Abbreviations}
\label{s:Definitions_Acronyms_Abbreviations}%

\section{Revision history}
\label{s:Revision_history}%


\section{Reference Documents}
\label{s:Reference_documents}%

\section{Document Structure}
\label{s:Document_Structure}%

