\section{Product perspective}
\label{s:Product_perspective}%

\subsection{Class Diagram}
\label{ss:class_diagram}%
The diagram below represent the classes of the system and the relationships between them.

\subsection{State Diagrams}
\label{ss:state_diagrams}%

\subsection{Scenarios}
\label{ss:scenarios}%
\textbf{Scenario 1:} Professor Harry is a professor teaching at Politecnico di Milano together with professor Donald. Harry would like to encourage his students to study during the course, instead of having to study everything a few days before the exam. To do so he came up with the idea to create a challenge where the students can test their preparation and earn some extra points in the exam. While talking with other colleagues, Prof. Harry discovered CKB and he thought it was the perfect fit to implement his idea. The first thing he does is to go to the webpage of CKB and create an account by clicking the \textbf{\textit{Sign up}} button and providing some information about himself. Afterwards he is redirected to the home page of the platform where he can click the button \textbf{\textit{Create Competition}}, and finally he inserts the name of the competition and the subscription deadline. At this point he wants to invite his colleague Donald to manage the competition with him; since he is an ED in the competition he can click on \textbf{\textit{Invite Educator}} in the competition page, then provides the email of Donald's account, who will be part of the competition once he accepts the invite.


\textbf{Scenario 2:} Professor Harry is an ED of a competition, within which he wants to create a battle. To do so he enters the dashboard of the competition, clicks on the button \textbf{\textit{Create Battle}} and provides everything the platform needs: description, test cases, build automation scripts, deadlines, accepted sizes of groups. Marco, a ST who subscribed to the competition, received the notitication about the newly created battle via email. Outside of the platform Marco agreed with a couple of friends to participate in the battle together. Marco then goes on the competition page and finds the newly created battle, here he finds two buttons, \textbf{\textit{Participate as: 1. Loner 2. Team}}; he clicks the second button to participate as a team and invites his friends by providing the platform his friends' account email. Once Marco's friends accepted the invite the subscription to the battle will be automatically finalized by the platform.


\textbf{Scenario 3:} Professor Harry wants to give credit to the hardest working student, so while creating the competition he decided to create a new badge. The hardest working ST is the one that has written the highest amount of code lines among all the battles in the same competition. To implement this badge Prof. Harry must create a new variable \textbf{\textit{hardest\_worker}} and provide the code that defines how to compute the value of such variable. Some time after the specification of this new badge, a ST, participating in the competition and in the current battle, called Marco, pushes a commit to his repository. Since all students are supposed to setup \textit{GitHub Actions}, CKB is notified about Marco's commit, so it proceeds to run the required processes to calculate the new score, but also checks if Marco acquired new badges by checking their rules. Assuming that with the last commit Marco has now the most written lines of code, CKB assigns to him the \textit{hardest\_worker} badge.


\textbf{Scenario 4:} Marco and his team participated in a battle provided by Prof. Harry, one of the ED of the competition. Since the battle ends the next day, Marco wants to look at the partial rankings of the battle, so he goes on the page related to the battle and clicks on the \textit{Results} section, and sees his team at the bottom of the chart. Understandably, Marco's team resumes to work on the problem and they are able to commit a new version of their solution, which increased their placement in the partial rankings of the battle. The submission deadline now expired and the EDs now want to manually check the work of their STs to assign manually a score to each team; to do this Prof. Harry goes on the battle page and clicks on \textbf{\textit{Perform Manual check}}, which will redirect the ED to another page where he can inspect the source code of each team and give a score to each final work. Once this consolidation phase has been declared finished by an ED, CKB sends to all the STs subscribed to the competition a notification that the battle's results are available and the global scores of the competition have been updated.


\section{Product functions}
\label{s:Product_functions}%

\section{User characteristics}
\label{s:User_characteristics}%


\section{Assumptions, dependencies and constraints}
\label{s:Assumptions_dependencies_and_constraints}%


