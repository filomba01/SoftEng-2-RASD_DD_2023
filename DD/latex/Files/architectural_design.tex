\section{Overview: High-level components and their interaction}
\label{s:overview}%

\section{Component view}
\label{s:component-view}%
\subsection*{Client}
\begin{itemize}
  \item \textbf{WebAppUI}: represents the web application, which is reachable by any browser and usable only after a user has been authenticated using the \textbf{\textit{AuthInterface}}. It allows both STs and EDs to perform a certain set of actions based on the type of user by using the \textbf{\textit{DashboardInterface}}
\end{itemize}
\subsection*{Server}
\begin{itemize}
  \item \textbf{AuthenticationService}: provides the set of procedures required to handle the authentication of a user into the system (i.e., login, registration)
  \item \textbf{DashboardManager}: used as an intermediary between the \textit{WebAppUI} and the other components of the server to provide the web application only the functionality strictly necessary for it to work properly.
  \item \textbf{CompetitionManager}: handler of all functionalities regarding competitions (for both STs and EDs) excluding badges, which management has been delegated to the \textit{BadgeMonitor} component
  \item \textbf{Battle Manager}: same as the \textit{CompetitionManager} but transposed to handle battles. Delegates the team creation/deletion to the \textit{TeamManager}
  \item \textbf{TeamManager}: used to manage the teams used by the STs to participate to battles, it also includes the invite handling
  \item \textbf{BadgeManager}: its purpose is to deal with badges, create/remove a badge with its related rule, and perform checks to verify if a ST enrolled in a competition satisfies any badge rule defined in such competition
  \item \textbf{NotificationService}: its main goal is to implement procedures to send various types of notifications to the application's users; to send such notifications (mails) it uses the \textbf{\textit{MailAPI}} provided by the \textit{MailServer}
  \item \textbf{DataManager}: mediator between the model components and the \textit{DBMS}; it uses the procedures provided by the \textbf{\textit{DBMS\_API}} to implement a set of functions, which have the sole purpose of manipulating the database or retrieving information from it
  \item \textbf{EvaluatorController}: it is called through the \textbf{\textit{EvaluationAPI}} by \textit{GitHub Actions} on each commit performed by a team. Its purpose is to control the evaluation process by calling the \textit{StaticAnalyzer} and the \textit{CodeEvaluator} to perform the proper checks on the last committed code. Moreover, it uses the functions provided by the \textbf{\textit{EvaluatorInterface}}, \textbf{\textit{AnalyzerInterface}}, \textbf{\textit{ScoreInterface}} to change the configuration of the code evaluator and the static analyzer, with the last interface it sets the score functions used by the \textit{PointManager}
  \item \textbf{CodeEvaluator}: used to execute the source code of a team's repository with the set of test cases provided by the EDs of the current battle. It exposes the \textbf{\textit{EvaluatorInterface}} to be used to configure the evaluator for instance in terms of test cases or maximum execution time
  \item \textbf{StaticAnalyzer}: provides the \textbf{\textit{AnalyzerInterface}} to configure the static analyzer, which will be used to analyze some input code. Such analysis will return some results, which can be sent to the \textit{PointManager} through the proper interface still provided by the PointManager (this holds also for the \textit{CodeEvaluator})
  \item \textbf{PointManager}: as the name suggests, it manages the points assignment. In particular it provides two interfaces that are used by the \textit{StaticAnalyzer} and the \textit{CodeEvaluator} to send the results of their analysis. Such results are put in a score function designed by the EDs to compute the partial score of a single analysis; once the \textit{PointManager} computes the score of both the evaluations (static analysis and code evaluation) it updates the final score of the team in the database

\end{itemize}

\begin{figure}[H]
  \begin{center}
    \includegraphics[width=\textwidth, keepaspectratio]{ComponentDiagramLow.png}
    \caption{Component diagram low level}
    \label{fig:component_diagram_low}
  \end{center}
 
\end{figure}
\section{Deployment view}
\label{s:deployment-view}%

\newpage

\section{Runtime view}
\label{s:runtime-view}%
This section contains the sequence diagrams of the most important operations of the system. The diagrams include the component that we have already described in the previous section and the external components that are involved in the operations.

\subsubsection*{User Registration}
\label{ss:registration_diagram}%
When a user wants to register to the system, he/she has to fill in the registration form and submit it. The difference between a ST and an ED is in the information passed with \textit{UserInfo} object, where the ED has to insert also the information about the institution he/she works for.

The whole process is mainly handled by the \textit{Authentication Manager} component, that interact with the \textit{Data Manager} component to validate the information and insert the new user into the DBMS.

The system will check if the information inserted are valid and if the user is not already registered. This check is done internally from CKB and if the information are valid and the user is not already registered, the system will insert it into the DBMS and sends an email to the user with a link to confirm the registration, using the \textit{Notification Manager} component. The user will click on the link and the \textit{Authentication Manager} will confirm the registration.

If the information inserted are not valid, the user is already registered or the confirmation link is expired, the system will send an error message to the user.

\begin{figure}[H]
  \centering
  \includegraphics[width=\textwidth,height=\textheight, keepaspectratio]{SequenceDiagrams/01-UserRegistration.png}
  \caption{Registration sequence diagram}
  \label{fig:registratio_diagramn}
\end{figure}

\subsubsection*{User Login}
\label{ss:login_diagram}%
When a user wants to login to the system, he/she has to fill in the login form and submit it. This process is equal for both the ST and the ED. As the User Registration, the whole process is handled by the \textit{Authentication Manager} component, that interact with the \textit{Data Manager} component to validate the information. Once the user is logged in, the \textit{Authentication Manager} will generate a token for the user, send it to the client and the user can finally access the dashboard.

If the login information are not valid, the system will show an error message to the user.

\begin{figure}[H]
  \centering
  \includegraphics[width=\textwidth,height=0.7\textheight, keepaspectratio]{SequenceDiagrams/02-UserLogin.png}
  \caption{Login sequence diagram}
  \label{fig:login_diagramn}
\end{figure}

\subsubsection*{ED Creates Competition}
\label{ss:create_competition_diagram}
Here is shown how a competition is created by an ED. The ED has to fill in the form with the information about the competition and submit it. The \textit{Competition Manager} component will check if the name is available and if so the competition will be created and inserted into the DBMS using the \textit{Data Manager} component, otherwise a new name will be requested. The system will then returns the competition and will be shown the competition page.

\begin{figure}[H]
  \centering
  \includegraphics[width=\textwidth,height=0.74\textheight, keepaspectratio]{SequenceDiagrams/04-EducatorCreatesCompetition.png}
  \caption{Create competition sequence diagram}
  \label{fig:create_competition_diagramn}
\end{figure}

\subsubsection*{ST Joins Competition}
\label{ss:join_competition_diagram}
The ST firstly visualize all the available competitions in the platform, then he/she can choose one of them and join it. The \textit{Competition Manager} component will handle both the search of the available competitions and the insertion into the DBMS using the \textit{Data Manager} component. The system will then show a success message to the user.

\begin{figure}[H]
  \centering
  \includegraphics[width=\textwidth,height=\textheight, keepaspectratio]{SequenceDiagrams/05-StudentJoinCompetition.png}
  \caption{Join competition sequence diagram}
  \label{fig:join_competition_diagramn}
\end{figure}

\subsection*{ED Creates Battle}
\label{ss:create_battle_diagram}
When a ED wants to create a battle, he/she needs to be in a competition page that he/she manages. The interaction is divided in two parts: the first one is where the ED inserts the general information about the battle, the name is then validated by the \textit{Battle Manager} component and if it is valid the system will show the second part of the form where the ED can insert the information about the automatic evaluation and static analysis. The \textit{Battle Manager} component will check if the information are valid and if so the battle will be created and inserted into the DBMS using the \textit{Data Manager} component.

If the battle is successfully created, the system will send a notification, through the \textit{Notification Manager} to all the ST enrolled in the competition where the battle has been created.

\begin{figure}[H]
  \centering
  \includegraphics[width=\textwidth,height=0.83\textheight, keepaspectratio]{SequenceDiagrams/06-EducatorCreatesBattle.png}
  \caption{Create battle sequence diagram}
  \label{fig:create_battle_diagramn}
\end{figure}

%9
\subsection*{ST Joins Battle}
To be able to join a battle, a \textit{ST} must be enrolled in a competition and in such competition there has to be at least 1 battle available (subscription deadline not expired). Assuming all those things, when a ST wants to join a battle, he/she has two options: create a team (with the possibility, if enabled by the EDs of the battle, to be a singleton) or to join an already existing team. In the first case the ST has to provide the system all the information needed for the team registration, and if the team has been created correctly he/she can invite other STs to join his/her team (if the ST, owner of the team, wants to participate alone he/she must skip the invite part). In the latter case instead, the ST looks at the list of existing and available teams and if he/she wants to join one of those team they can simply click on the join button. \\ \\
Since the sequence diagram below was pretty long we decided to split it in two so that it was easier to read it; this means that the two following diagrams are to be understood as consecutive (the [\textit{alt 'ST wants to join an existing public T'}] fragment, in the second part of the diagram, is the [\textit{else}] of the [\textit{alt 'ST selects to create a new T'}] fragment in the first part of the diagram)


\begin{figure}[H]
  \centering
  \includegraphics[width=\textwidth,height=0.83\textheight, keepaspectratio]{SequenceDiagrams/09.1-StudentJoinsBattle.png}
  \caption{Create battle sequence diagram}
  \label{fig:st_joins_battle_1}
\end{figure}

\begin{figure}[H]
  \centering
  \includegraphics[width=\textwidth, keepaspectratio]{SequenceDiagrams/09.2-StudentJoinsBattle.png}
  \caption{Create battle sequence diagram}
  \label{fig:st_joins_battle_2}
\end{figure}

%10
\subsection*{User Visualizes battle Rankings}
Assuming that a User (ST or ED) is enrolled firstly in a competition and secondly subscribed to a battle that is still ongoing, such User can see the partial leaderboard of the battle. The sequence diagram below describe exactly this process, starting from the page of the competition. 
\begin{figure}[H]
  \centering
  \includegraphics[width=\textwidth,height=0.75\textheight, keepaspectratio]{SequenceDiagrams/10-BattleRankingVisualization.png}
  \caption{Create battle sequence diagram}
  \label{fig:user_visualize_rankings}
\end{figure}

%11
\subsection*{ED Evaluates Code}

%12
\subsection*{ST Accepts invite}

%13
\subsection*{ST Visualizes other ST Profile}

\section{Component interfaces}
\label{s:component-interfaces}%

\section{Selected architectural styles and patterns}
\label{s:selected-architectural-styles-and-patterns}%

\section{Other design decisions}
\label{s:other-design-decisions}%
