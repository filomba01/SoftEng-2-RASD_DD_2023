\section{Purpose}
\label{s:Purpose}%

This document contains the design description of the \emph{CodeKataBattle} system. It includes the architectural design, the user interface design and the descrpition of all the operations that the system will perform. It also show how the requirements and use cases detailed in the RASD document are satisfied by the design of the system.

This document is intended to be read by the developers of the system, the testers and the project managers. It is also intended to be used as a reference for the future maintenance of the system.

\section{Scope}
\label{s:Scope}%

The \emph{CodeKataBattle} system is a web application that allows educators to create challenges for their students based on solving programming problems. In particular the system is based on the concept of \emph{Code Kata} that is an exercise in programming which helps a programmer hone their skills through practice and repetition. The system will allow the educators to create competition and battle based on \emph{Code Kata}. The students will be able to participate in the battles with a team or by themselves and solve the challenges in order to earn points. The system will also provide a leaderboard that will show the ranking of the students based on their scores.

A more detailed description of the system can be found in the RASD document, whilist in this document is provided a detailed description of the design of the system to implement the requirements and use cases described in the RASD document.

\newpage

\section{Definitions, Acronyms, Abbreviations}
\label{s:definitions-acronyms-abbreviations}%
\subsection{Definitions}
\label{ss:Definitions}

\begin{table}[H]
  \begin{tabular}{|l|p{0.8\textwidth}|}

    \hline
    User & Anyone interacting with the system, it can be both a Student or an Educator    \\
    \hline
    Manage & Create, supervise and edit a certain element of the application. \\
    \hline
    Code Kata & A challenge intended to improve programming abilities, including description, test cases and build automation scripts. \\
    \hline
  \end{tabular}
  \caption{List of definitions}
  \label{tab:definitions}
\end{table}

\subsection{Acronyms}
\label{ss:Acronyms}

\begin{table}[H]
  \begin{tabular}{|l|l|}

    \hline
    ST & Student \\
    \hline
    ED & Educator \\
    \hline
    CKB & CodaKataBattle \\
    \hline
    RASD & Requirements Analysis and Specification Document     \\
    \hline
    SAT & Static Analyzer Tool    \\
    \hline
    T & Team    \\
    \hline
    MVC & Model View Controller    \\
    \hline
  \end{tabular}
  \caption{List of Acronyms}
  \label{tab:acronyms}
\end{table}


\section{Revision history}
\label{s:revision-history}%

\begin{table}[H]
  \begin{tabular}{|l|l|l|}

    \hline
    Date & Revision & Notes    \\
    \hline
    07/01/2024 & v1.0 & First release    \\
    \hline

  \end{tabular}
  \caption{Revision table}
  \label{tab:revision}
\end{table}

\section{Reference Documents}
\label{s:reference-documents}%

\begin{itemize}
  \item The specification document of the project: \textit{Assignment RDD AY 2023-2024}
  \item The RASD document of the project
\end{itemize}


\section{Document Structure}
\label{s:document-structure}%

The document is divided in 6 sections:
\begin{itemize}
  \item \textbf{Introduction}: provides a brief description of the purpose and the scope of the system. Morover it contains the definitions, acronyms and abbreviations used in the document and the reference documents.
  \item \textbf{Architectural Design}: this section provides a description of the architecture of the system, including the components and the interfaces between them. It also includes the runtime view of the most important operations of the system, the deployment view and the architectural styles and patterns used in the system. 
  \item \textbf{User Interface Design}: includes the mockups of the user interface of the system.
  \item \textbf{Requirements Traceability}: this section shows how the requirements described in the RASD document are satisfied by the design of the system. 
  \item \textbf{Implementation, Integration and Test Plan}: this includes the step-by-step plan for the implementation and testing of the system.
  \item \textbf{Effort Spent}: this section highlights the effort spent by each member of the group to redact this document.
\end{itemize}
